
\documentclass[a4paper,11pt]{article}

\usepackage{latexsym}
\usepackage[empty]{fullpage}
\usepackage{titlesec}
\usepackage{marvosym}
\usepackage[usenames,dvipsnames]{color}
\usepackage{verbatim}
\usepackage{enumitem}
\usepackage[pdftex]{hyperref}
\usepackage{fancyhdr}
\usepackage{graphicx}
\usepackage{wrapfig}

\pagestyle{fancy}
\fancyhf{} % clear all header and footer fields
\fancyfoot{}
\renewcommand{\headrulewidth}{0pt}
\renewcommand{\footrulewidth}{0pt}

% Adjust margins
\addtolength{\oddsidemargin}{-0.375in}
\addtolength{\evensidemargin}{-0.375in}
\addtolength{\textwidth}{1in}
\addtolength{\topmargin}{-.5in}
\addtolength{\textheight}{1.0in}

\urlstyle{same}

\raggedbottom
\raggedright
\setlength{\tabcolsep}{0in}

% Sections formatting
\titleformat{\section}{
  \vspace{-4pt}\scshape\raggedright\large
}{}{0em}{}[\color{black}\titlerule \vspace{-5pt}]

%-------------------------
% Custom commands
\newcommand{\resumeItem}[2]{
  \item\small{
    \textbf{#1}{: #2 \vspace{-2pt}}
  }
}

\newcommand{\resumeSubheading}[4]{
  \vspace{-1pt}\item
    \begin{tabular*}{0.97\textwidth}{l@{\extracolsep{\fill}}r}
      \textbf{#1} & #2 \\
      \textit{\small#3} & \textit{\small #4} \\
    \end{tabular*}\vspace{-5pt}
}

\newcommand{\resumeSubItem}[2]{\resumeItem{#1}{#2}\vspace{-4pt}}

\renewcommand{\labelitemii}{$\circ$}

\newcommand{\resumeSubHeadingListStart}{\begin{itemize}[leftmargin=*]}
\newcommand{\resumeSubHeadingListEnd}{\end{itemize}}
\newcommand{\resumeItemListStart}{\begin{itemize}}
\newcommand{\resumeItemListEnd}{\end{itemize}\vspace{-5pt}}

%-------------------------------------------
%%%%%%  CV STARTS HERE  %%%%%%%%%%%%%%%%%%%%%%%%%%%%

\begin{document}

%----------HEADING-----------------
\begin{tabular*}{\textwidth}{l@{\extracolsep{\fill}}r}
  \textbf{\Large Mateusz Kapusta} & \href{mailto:mr.kapusta@student.uw.edu.pl}{mr.kapusta@student.uw.edu.pl} \\
  \href{https://github.com/Wesenheit}{github.com/Wesenheit} & +48 530 510 849 \\
\end{tabular*}

\section{Profile}
%\begin{wrapfigure}{L}{0.20\textwidth}
%  \vspace{-0.6cm}
%    \includegraphics[width=0.16\textwidth]{MK_photo.jpg}
%\end{wrapfigure}
%I am a scientific computing enthusiast with a strong foundation in mathematics, machine learning, and software engineering.
%My research has focused on analyzing large datasets and developing Bayesian models,
%resulting in multiple peer-reviewed publications in astronomical data analysis.
%My experience in developing deep-learning models resulted in a poster presented at the 
%national machine learning conference ML in PL 2024. I am eager to apply my skills to impactful, real-world problems.

 % Most of my work relies on writing fast and efficient code to process/simulate data.
%Mostly interested in the intersection of physics and computer science.
%Looking to apply my skills in a dynamic, problem-solving-oriented team. %, in the field of financial risk modelling.
%Looking to apply my skills in a new.
%Willing to relocate for the job.

I am a fifth-year master's student with a strong academic background in computer science, mathematics and physics,
complemented by research experience in astronomical data analysis. My interdisciplinary studies have provided me with a solid foundation
in applied mathematics, including machine learning, numerical methods, and stochastic processes.
In my research, I have focused on analyzing large datasets and developing Bayesian models to describe them, leading to few peer-reviewed publications in astronomical data analysis.
These projects have extensively utilized Python, providing me with strong software engineering skills.
My experience in developing deep-learning models resulted in a poster presented at the national machine learning conference ML in PL 2024.
%In my free time, I enjoy participating in or judging math and science competitions to keep my quantitative skills sharp.
Willing to work in-pearson in Wrocław office.


\section{Work Experience}
\resumeSubHeadingListStart

\resumeSubheading
{Research Assistant – Astronomical Observatory, Warsaw University}{July 2022 – Present}
%{\vspace{0.0cm}}{Python \& C}
{Python, Machine Learning and C}{}
%{Statistical Modeling \& Software Acceleration}{}
\resumeItemListStart
  \item Developed various statistical tools (ANNOVA, MCMC) in Python, with performance-critical components implemented in C using SIMD intrinsics.
  \item Analyzed massive astronomical datasets, cross-matching catalogs and selecting candidate objects using statistical filters.
  \item Built and tuned various Bayesian models to analyze astronomical data, performed inference with MCMC, contributing to model selection and validation.
 %   \item Applied data analysis and image processing methods to improve object detection on the state of the art astronomical detectors. 
  \item Research output includes few publications in the field of astronomical data analysis.
  \item Employed in NCN grant 2021/41/B/ST9/00252.
\resumeItemListEnd

\resumeSubheading
{C/C++ Intern – Polish Academy of Sciences}{July 2021 – July 2022}
{\vspace{-0.5cm}}{}
%{C/C++ Code Development}{}
\resumeItemListStart
  \item Ported legacy C++98 code to modern C++17 in part of the LORENE library. Enabled MPI task launching to scale calculations across multiple nodes of computational cluster.
  \item Ported slow MATLAB code to C++ with OpenMP parallelism for scalable performance. Reached $>30$ times speedup on modern desktop computers.
\resumeItemListEnd

\resumeSubHeadingListEnd

\section{Projects}
\resumeSubHeadingListStart
\resumeSubheading
{Computational Fluid Dynamics on GPU-s}{Oct 2024 – Present}
{\vspace{-0.5cm}}{}
\resumeItemListStart
  \item Built high-performance multi-node GPU accelerated simulation code to simulate relativistic fluids in both C++ and Julia.
  \item Developed custom CUDA kernels and performed extensive benchmarking with Nsight Compute to pinpoint exact bottlenecks of the application (compute - memory bandwidth - intercommunication).
  \item Supported by EURO-HPC-DEV-2025D02-085 development grant on the Italian supercomputer Leonardo, where I am PI.
\resumeItemListEnd


\resumeSubheading
{Deep Learning for Simulation Based Inference}{Oct 2023 – Present}
{\vspace{-0.5cm}}{}
{Deep Learning \& Bayesian Modeling}{}
\resumeItemListStart
  \item Developed a transformer-based neural network for Bayesian inference via Simulation-Based Inference (SBI), aiming to replace slow MCMC methods.
  \item Designed a custom hybrid architecture combining Set Invariant Transformers and Masked Autoregressive Flow.
  \item Model allows to compute fast approximation of the posterior for any graphical model with set-like observed variables.
  \item Implemented model in PyTorch and tracked experiments using Neptune.ai.
  \item Final model reached $4$ orders of magnitude speedup vs MCMC for usual use-cases.
  \item Work presented as a poster at ML in PL 2024; a peer-reviewed publication is in preparation.
\resumeItemListEnd

%\resumeSubheading
%{Clustering with Gaussian Mixture Variational Autoencoder}{December 2022 – May 2023}
%{\vspace{-0.5cm}}{}
%{Deep Learning \& Bayesian Modeling}{}
%\resumeItemListStart
%  \item Implemented Gaussian Mixture Variational Autoencoder in the PyTorch framework, which was used to cluster the gene expression data.
%  \item Benchmarked the clustering efficiency for different versions of the model.
%  \item Results can be found in the blog post \href{https://wesenheit.github.io/bayes/2023/03/19/GMVAE-clustering-applied-to-RNA-sequencing.html}{https://wesenheit.github.io/bayes/2023/03/19/GMVAE-clustering-applied-to-RNA-sequencing.html}.
%\resumeItemListEnd

\resumeSubHeadingListEnd

\section{Programming Skills}
\resumeSubHeadingListStart
  \item \textbf{Languages:} Python (advanced), C/C++ (advanced), Julia (advanced),  SQL (intermediate), R (basic), Rust (basic), Excel.
  \item \textbf{Python frameworks/libraries:} Numpy, Pandas, PyTorch, TensorFlow, JAX/Flax, scikit-learn, Pyro, PyMC, BlackJAX, XGBoost, Matplotlib,
  \item \textbf{Tools:} MPI, OpenMP, CUDA, SIMD, Git, Bash, Linux (expert-level)
  \item \textbf{DevOps \& Workflow:} Neptune.ai, Docker, CI/CD, CMake.
\resumeSubHeadingListEnd

\section{Awards}
\resumeSubHeadingListStart
  \item Rector's Scholarship: 2020–2025 (awarded 5 consecutive years)
  \item Gold Medal – University Physics Competition 2021 (team)
  \item Finalist – Polish Physics Olympiad (67th, 69th - 10th place)
  \item Finalist – Polish Mathematical Olympiad (70th)
  \item Winner – 62nd Polish Astronomical Olympiad
  \item Bronze Medal – 11th International Olympiad on Astronomy and Astrophysics
  \item Silver Medal – 1st Global e-Competition on Astronomy and Astrophysics (2020)
\resumeSubHeadingListEnd


\section{Education}
\resumeSubHeadingListStart

\resumeSubheading
{University of Warsaw – MSc in Physics \& Computer Science}{Warsaw, Poland}
{Faculty of Physics; GPA: 4.88/5}{Oct 2023 –  September 2025 (expected)}

\resumeSubheading
{University of Warsaw – BSc in Physics \& Computer Science}{Warsaw, Poland}
{Inter-faculty Individual Studies in Mathematics and Natural Sciences; GPA: 4.96/5}{Oct 2020 – Jul 2023}
\par\vspace{0.2cm}
Interdisciplinary curriculum across the Faculty of Physics (FUW) and Faculty of Mathematics, Informatics, and Mechanics (MIMUW).
Specialization in Machine Learning, Data Analysis, High Performace Computing.

\resumeSubheading
{III High School}{Wrocław, Poland}
{Graduated with Distinction; GPA: 4.8/5}{Sep 2017 – Jul 2020}
\resumeSubHeadingListEnd

%\section{Extracurricular Course Highlights}
%\resumeSubHeadingListStart
%  \item \textbf{Software Development:} Numerical Methods (low-level C, signal processing), High-Performance Computing (C++, CUDA, MPI, performance analysis).
%  \item \textbf{Machine Learning:}  Statistical Data Analysis I (R,
%  statistical learning), Statistical Data Analysis II (Bayesian modeling, deep learning), Deep Neural Networks (Computer Vision, Natural Language Processing, Reinforcement learning)
%\resumeSubHeadingListEnd
\section{Extracurriculars}
  \resumeSubHeadingListStart
  \item Active member of the Statistical Journal Club at the Astronomical Observatory, presenting cutting-edge research on statistics and deep learning in astronomy.
  \item Jury at various science competitions including "16th International Olympiad on Astronomy and Astrophysics" and Naboj math competition (2023, 2024, 2025). 
\resumeSubHeadingListEnd

\section{Posters}
\resumeSubHeadingListStart
  \item \textbf{Kapusta, M.} “Iris-ML: Neural Posterior Estimation for Spectral Energy Distribution Fitting.” Poster presented at ML in PL 2024.
%  \item Mróz, P., Udalski, A., Szymański, M., \textbf{Kapusta, M.}, et al. “Microlensing Optical Depth and Event Rate Toward the LMC from 20 Years of OGLE Observations.” ApJS, 273, 4 (2024) \href{https://arxiv.org/abs/2403.02398}{arXiv:2403.02398}
%  \item \textbf{Kapusta, M.}, Mróz, P. “The Search for Dormant Black Holes in OGLE Data.” Acta Astron., 73, 197 (2023) \href{https://arxiv.org/abs/2401.11293}{arXiv:2401.11293}
\resumeSubHeadingListEnd

\section{Languages}
\begin{itemize}
  \item English – C1 (TOEFL 103)
  \item Polish – Native
  \item German – A2/B1
\end{itemize}
\end{document}
