
\documentclass[a4paper,11pt]{article}

\usepackage{latexsym}
\usepackage[empty]{fullpage}
\usepackage{titlesec}
\usepackage{marvosym}
\usepackage[usenames,dvipsnames]{color}
\usepackage{verbatim}
\usepackage{enumitem}
\usepackage[pdftex]{hyperref}
\usepackage{fancyhdr}


\pagestyle{fancy}
\fancyhf{} % clear all header and footer fields
\fancyfoot{}
\renewcommand{\headrulewidth}{0pt}
\renewcommand{\footrulewidth}{0pt}

% Adjust margins
\addtolength{\oddsidemargin}{-0.375in}
\addtolength{\evensidemargin}{-0.375in}
\addtolength{\textwidth}{1in}
\addtolength{\topmargin}{-.5in}
\addtolength{\textheight}{1.0in}

\urlstyle{same}

\raggedbottom
\raggedright
\setlength{\tabcolsep}{0in}

% Sections formatting
\titleformat{\section}{
  \vspace{-4pt}\scshape\raggedright\large
}{}{0em}{}[\color{black}\titlerule \vspace{-5pt}]

%-------------------------
% Custom commands
\newcommand{\resumeItem}[2]{
  \item\small{
    \textbf{#1}{: #2 \vspace{-2pt}}
  }
}

\newcommand{\resumeSubheading}[4]{
  \vspace{-1pt}\item
    \begin{tabular*}{0.97\textwidth}{l@{\extracolsep{\fill}}r}
      \textbf{#1} & #2 \\
      \textit{\small#3} & \textit{\small #4} \\
    \end{tabular*}\vspace{-5pt}
}

\newcommand{\resumeSubItem}[2]{\resumeItem{#1}{#2}\vspace{-4pt}}

\renewcommand{\labelitemii}{$\circ$}

\newcommand{\resumeSubHeadingListStart}{\begin{itemize}[leftmargin=*]}
\newcommand{\resumeSubHeadingListEnd}{\end{itemize}}
\newcommand{\resumeItemListStart}{\begin{itemize}}
\newcommand{\resumeItemListEnd}{\end{itemize}\vspace{-5pt}}

%-------------------------------------------
%%%%%%  CV STARTS HERE  %%%%%%%%%%%%%%%%%%%%%%%%%%%%


\begin{document}

%----------HEADING-----------------
\begin{tabular*}{\textwidth}{l@{\extracolsep{\fill}}r}
  \textbf{\Large Mateusz Kapusta} & Email : \href{mailto:mr.kapusta@student.uw.edu.pl}{mr.kapusta@student.uw.edu.pl}\\
  https://github.com/Wesenheit & Mobile : +48 530 510 849\\
\end{tabular*}
\section{Profile} 
I am a fifth-year Master's student at the Faculty of Physics (and partially at the Faculty of Mathematics, Informatics and Mechanics), University of Warsaw. My main interest lies in the 
modern data analysis (mostly Bayesian modeling), mathematical modeling, and statistics. Most of my education
was supplemented with the relevant research experience in the field of natural science, where I could apply my knowledge to model real-life data
and master problem-solving skills.
Additionally, I am also interested in the low-level programming in C/C++, which I am mainly using to write codes, which are more demanding in terms of performance.
I enjoy tackling different problems from various fields and have a long track of relevant projects, which I publish on my GitHub account
https://github.com/Wesenheit.

%-----------EDUCATION-----------------
\section{Education}
  \resumeSubHeadingListStart
  \resumeSubheading
  {Warsaw University - major in Astrophysics}{Warsaw, Poland}
  {\footnotesize Faculty of Physics (Msc);  GPA: 4.88/5}{Oct. 2023 -- Jul. 2025 (expected)}
  \resumeItemListStart
    \resumeItem{HPC \& acceleration}{
      Designing CUDA + MPI accelerated multi-node numerical code for the relativistic hydrodynamics in Julia, aiming at petascale-size systems. Utilizing profiling tools like Nsight-Compute 
      to pinpoint the bottlenecks of the application (bandwidth - compute - communication). Most of the work was centered around writing custom CUDA kernels.
      Performing aggressive optimization to run as fast as possible on an Ampere-based Nvidia GPU-s
      on several nodes of the supercomputer.
    }
    \resumeItem{Deep Learning}
    {
      Working on the transformer-based neural net for the astronomical data analysis. The model utilizes the Simulation-Based Inference (SBI)
      to replace MCMC-based models and accelerate the inference for modern astronomical datasets. During the project, I designed my own hybrid model
      (Set Invariant Transformer + Masked Autoregressive Flow) and 
      implemented it with the PyTorch library. During the project almost everything was implemented and trained from scratch.
      Obtained results were presented as a poster
      at national-level conference ML in PL 2024, and will be presented in peer-reviewed publication in the future.
    }
    \resumeItemListEnd
    \resumeSubheading
      {Warsaw University - major in Astrophysics (minor in Computer Science)}{Warsaw, Poland}
      {\footnotesize Inter-faculty Individual Studies in Mathematics and Natural Sciences (Bsc);  GPA: 4.96/5}{Oct. 2020 -- Jul. 2023}
      \par
      \vspace{0.2cm}
      Pursuing interdisciplinary studies, divided between Faculty of Physics (FUW) and Faculty of Mathematics, Informatics and Mechanics (MIMUW).
      \resumeItemListStart
        \resumeItem{C/C++ programming}{
            Working on MPI task launching with LORENE library. Porting some routines from old C++98 standard to utilize some of 
            the modern C++11 standard. Porting old matlab code to C++ with OpenMP parallelization, allowing to scale calculations on a computer cluster. 
        }
        \resumeItem{Data Analysis \& acceleration}{Working with astronomical databases, data cleaning and analysis.
          Creating statistical software, implementing various statistical methods (MCMC, ANOVA, and more).
          Using ADQL queries to analyse enourmous amounts of astronomical data.
          Utilizing low-level C programming with SIMD instructions to speed up the processing of time-series data, writing Python wrappers with Python C-API.
          During my studies I obtained basic research experience, most of my work was supported by NCN grant 2021/41/B/ST9/00252.
        }
        \resumeItem{Bioinformatics \& Statistics}{
          Completed courses from the field of statistical data analysis in the context of biological data, taught by the Computational Medicine group at
          Faculty of Mathematics, Informatics and Mechanics.
          Learned to use methods ranging from statistical ones (Random Forest, Linear Models, KNN, etc) through Bayesian ones (MCMC, VI, GMM) to deep learning (VAE),
          to model biological data.
        }
        \resumeItem{Deep Learning}{
          Attended courses designed for the Machine Learning Master's program (ar MIMUW), in the field of deep learning. Courses covered not only basic material from various 
          fields like Computer Vision, Natural Language Processing, and Reinforcement Learning, but also more advanced ones including modern state-of-the-art architectures 
          like transformers, rainbow agents, advanced detection models like FCOS, and more. Moreover, I gained expert-level knowledge from the field of Bayesian modelling 
          including traditional methods (MCMC, VI) and hybrid ones like Variational Autoencoders, Normalizing Flows and more.
        }
    \resumeItemListEnd
    \resumeSubheading
      {IIIrd High School}{Wroclaw , Poland}
      {\footnotesize Graduated with distinction;  GPA: 4.8/5}{Sep. 2017 -- July. 2020}
  \resumeSubHeadingListEnd

\section{Relevant university education}
  \resumeSubHeadingListStart
    \item \textbf{Programming courses}: Programming I (Python), Programming II  (C++), High Performance Computing (C++, CUDA, MPI, 
          performance analysis, {\it ongoing course}), Numerical Methods (performance benchmarking, C).
    \item \textbf{Advanced statistical lectures}: Statistical Data Analysis I  (R, machine learning, data processing), 
                                    Statistical Data Analysis II  (Bayesian modeling, deep learning, genomics),
                                    Deep Neural Networks  (NLP, CV, RL),
                                    Models of Applied Mathematics  (game theory, markov chains),
                                    Stochastic Processes (stochastic processes, time series)
    \item \textbf{Extracurricular activities}: Attending Statistical Journal Club (SJC) at the Astronomical Observatory, University of Warsaw. Active contribution to the journal club,
    presenting new computational advances at the intersection of Astronomy and Statistics/Deep Learning.
  \resumeSubHeadingListEnd

\section{Programming Skills}
  \resumeSubHeadingListStart
    \item{
      \textbf{Languages}{: Python (advanced), C/C++ (advanced), Julia (advanced), R (beginner), Rust (beginner)}
      \hfill \\
      \textbf{Data-analysis oriented libraries}{: pandas, jupyter, matplotlib, seaborn, SQL.}
      \hfill \\
      \textbf{ML libraries}{: Torch, TensorFlow, Jax/Flax/Haiku, scikit-learn, Pyro, PyMc, blackjax}
    }
    \item{
     \textbf{Additional skils}{: git, bash, very good knowledge of Linux OS.}
    }
  \resumeSubHeadingListEnd


\section{Awards}
    \resumeSubHeadingListStart
    \item Rector scholarship in academic year 2020/2021, 2021/2022, 2022/2023, 2023/2024 and 2024/2025.
    \item University Physics Competition 2021: Part of Faculty of Physics team wining gold medal
    \item Finalist of 69th and 67th Polish Physics Olympiad
    \item Finalist of 70th Polish Mathematical Olympiad
    \item Winner of 62th Polish Astronomical Olympiad
    \item Bronze Medalist at 11th International Olympiad on Astronomy and Astrophysics
    \item Silver medal at 1st Global e-Competition on Astronomy and Astrophysics (in
    place of 14th International Olympiad on Astronomy and Astrophysics), 2020
    \resumeSubHeadingListEnd
  
\section{Language}
\begin{itemize}
    \item English - C1 (103 TOEFL)
    \item Polish - native speaker
    \item German - A2/B1
\end{itemize}

%-------------------------------------------
\end{document}
