
\documentclass[margin, 10pt]{res} % Use the res.cls style, the font size can be changed to 11pt or 12pt here

\usepackage{helvet} % Default font is the helvetica postscript font
%\usepackage{newcent} % To change the default font to the new century schoolbook postscript font uncomment this line and comment the one above
\usepackage{fontawesome}% <=============================================
\usepackage{hyperref}
\usepackage{wrapfig}
\usepackage{graphicx}
\setlength{\textwidth}{5.5in} % Text width of the document

\begin{document}

%----------------------------------------------------------------------------------------
%	NAME AND ADDRESS SECTION
%----------------------------------------------------------------------------------------

\moveleft.5\hoffset\centerline{\large\bf Mateusz Kapusta} % Your name at the top
 
\moveleft\hoffset\vbox{\hrule width\resumewidth height 1pt}\smallskip % Horizontal line after name; adjust line thickness by changing the '1pt'
 
\moveleft.5\hoffset\centerline{Astronomical Observatory, University of Warsaw, Al. Ujazdowskie 4, 00-478 Warsaw,Poland}
\moveleft.5\hoffset\centerline{\href{https://github.com/Wesenheit}{\faGithubSquare https://github.com/Wesenheit}}
%\moveleft.5\hoffset\centerline{\href{https://wesenheit.github.io/}{Blog website}}
%----------------------------------------------------------------------------------------

\begin{resume}

\section{PROFILE}
\begin{wrapfigure}{1}{0.4\textwidth}
    \includegraphics[width = 0.35\textwidth]{self1.jpg}
\end{wrapfigure}
I am a Master's student currently studying at the Faculty of Physics, University of Warsaw. I completed my Bachelor's thesis under supervision of  Przemysław Mróz at the 
Warsaw University Observatory. I graduated from the collage of Inter-Faculty Individual Studies, where I had an opportunity to take different courses including 
specialized lectures for Mathematics and Computer Science majors.
During my studies I had an opportunity to work on different aspects of Astrophysics, ranging from observational projects to 
theoretical ones. Currently I work with Krzysztof Nalewajko (Nicolaus Copernicus Astronomical Center - NCAC), investigating astrophysical jets launched from 
Supermassive Black Holes with the help of numerical simulations.
In free time I like to explore Machine Learning based methods applied in the field of Natural Sciences,
solving problems across different domains from Bioinformatics to Astronomy.
%----------------------------------------------------------------------------------------
%	OBJECTIVE SECTION
%----------------------------------------------------------------------------------------
 
%----------------------------------------------------------------------------------------
%	EDUCATION SECTION
%----------------------------------------------------------------------------------------
\par
\vspace{1cm}
\section{EDUCATION}

{\sl Master of Science, \textbf{2023-2025} (expected)}\\
Master of Science in Astronomy\\
Warsaw University Observatory, Faculty of Physics\\

{\sl Bachelor of Science, \textbf{2020-2023}}\\
Inter-faculty Individual Studies in Mathematics and Natural Sciences, University of Warsaw, Poland\\
Grade:  4.96/5 (2-5 scale, 5 is highest), graduated with distinctions\\
Major: Astronomy, Physics\\
Minor: Mathematics, Computer Science \par
\vspace{0,3cm}
{\sl High School, \textbf{2017-2020}}\\
IIIrd Secondary School, Wroclaw, Poland\\
%graduation date: 2020\\
grade: 5.00 (1-6) scale\\
graduated with distinctions

\section{EXPERIENCE}

{\sl Student research assistant, Nicolaus Copernicus Astronomical Center}\hfill April 2023 - present\\
Working under supervision of Krzysztof Nalewajko (NCAC).
Project in cooperation with Bart Ripperda (CITA, Toronto) and Alexander Philippov (University of Maryland).
\begin{itemize}
    \item Working with extreme resolution Magnetohydrodynamical (MHD) simulations created with H-AMR code.
    \item Studying stability of a astrophysical jets launched from black holes.
    \item Investigating influence of magnetic flux eruptions, associated with Magnetically Arrested Discs (MAD), on the properties of the jet. 
\end{itemize} 


{\sl Student research assistant at Astronomical Observatory} \hfil November 2022 - present\\
Student position in grant 2021/41/B/ST9/00252, working under the supervision of Przemysław Mróz.
\begin{itemize}
    \item Performing MCMC modelling of microlensing events discovered as the part of $4$th phase of the OGLE project.
    \item Investigating Free Floating Planet (FFP) microlensing event OGLE-2023-BLG-0524. Working on theoretical modelling, analysing legacy Hubble Space Telescope (HST)
    photometry, performing detectability simulations in order to verify FFP hypothesis.
    \item Research resulted in 4th-author publication in ApJS, currently preparing first author publication about OGLE-2023-BLG-0524 event.
\end{itemize}



{\sl Bachelor's Thesis at Warsaw University Astronomical Observatory} \hspace*{0.1cm} July 2022 - July 2023\\
Working under the supervision of dr Przemysław Mróz on the data analysis from the OGLE survey to search for Dormant Black Hole candidates.
\begin{itemize}
    \item Analysing OGLE data using the method introduced in Gomel et al. \href{https://arxiv.org/abs/2008.11209}{2021} 
    \item Designing Python based MCMC code to assemble spectral energy distribution (SED) for candidate objects.
    \item Inference of the parameters of binaries using the OGLE and Gaia DR3 data, searching for compact companion stars.
    \item Project resulted in first-author paper accepted in Acta Astronomica.
\end{itemize}



{\sl Intern at Nicolaus Copernicus Astronomical Center}\hfill July 2022 - October 2022\\
Project: "Measuring the structure of relativistic jets in numerical simulation results" under the supervision of Krzysztof Nalewajko (NCAC) and prof. Agnieszka Janiuk (CFT PAN).
\begin{itemize}
    \item Worked with results from HARM MHD code to study the structure of magnetically arrested discs.
    \item Developed a few Python routines to search for magnetic reconnection and other interesting magnetic phenomena.
    \item Work accomplished during the internship resulted in second-author publication submitted to Astronomy \& Astrophysics (in revision).
\end{itemize}

{\sl Intern at Nicolaus Copernicus Astronomical Center}\hfill July 2021 - October 2023\\
Project: "Energy of a Strange Quark Star" under the supervision of Fatemeh Kayanikhoo  and dr M. Cemeljic
\begin{itemize}
    \item Worked with LORENE library to study the structure of relativistic strange quark stars, ported part of functions to work with 
    C++17 standard and MPI multithread environment.
    \item Developed Python code to calculate the external energy of a star contained in a magnetic field.
    \item Developed multi-threaded C++ code to calculate the equation of state of the magnetized strange matter.
\end{itemize}

%{\sl Workshop at Faculty of Physics Warsaw University}\hfill Spring 2020\\
%Project: "Collisions of ultra-cold particles" under the supervision of dr hab. Michał Tomza
%\begin{itemize}
%    \item Developed code (in Python) solving the Schrodinger equation with DVR (discrete variable representation) method. 
%\end{itemize}


%{\sl Workshop at Institute of Physics of the Polish Academy of Sciences}\hfill Winter 2019 \\
%Project: 'Selected magnetic characteristics of solid body' under the supervision of dr hab. prof. IF PAN Andrzej Łusakowski

%\begin{itemize} \itemsep -2pt % Reduce space between items
%    \item Developed Monte Carlo simulations (in C++) of two basic lattice models of statistical mechanics: Ising model and XY model.
%    \item Research on phase transitions in those two models.
%\end{itemize}
    
    
    
    
    
    %----------------------------------------------------------------------------------------
    
    %---------------------------------------------------------------------------------------- 
    
\section{PUBLICATIONS}
    \begin{itemize}
        \item K. Nalewajko, \textbf{M. Kapusta}, A. Janiuk "Chaotic Magnetic Disconnections Trigger Flux Eruptions in Accretion Flows Channeled onto Magnetically Saturated Kerr Black Holes" 
        Accepted in Astronomy \& Astrophysics \texttt{[arXiv:2410.08280]}
        \item P.Mróz, A.Udalski, M.Szymański, \textbf{M. Kapusta}, et al. "Microlensing Optical Depth and Event Rate toward the Large Magellanic Cloud Based on 20 yr of OGLE Observations"
        Accepted in ApJS \texttt{[arXiv:2403.02398]}
        \item \textbf{M. Kapusta}, P. Mróz. "The search for Dormant Black Holes in the OGLE data"\\ Accepted in Acta Astronomica \texttt{[arXiv:2401.11293]}
        
        
        \item F. Kayanikhoo, \textbf{M. Kapusta}, M. Cemeljic.{"The maximum mass and deformation of rotating strange quark stars with strong magnetic fields"}\\
        Submitted to Physical Review D \texttt{[arXiv:2305.03055]}
    \end{itemize}

\section{TALKS \& POSTERS}
    \begin{itemize}
        \item \textbf{M. Kapusta} "Extreme resolution GRMHD simulations of Astrophysical jets", presentation at 10th Symposium for Young Researchers, presentation awarded with distinctions (September 2024, Warsaw)
        \item \textbf{M. Kapusta} "Iris-ML: Neural Density Estimation for the Spectral Energy Distribution fitting." poster at GHOST 2024 Machine learning conference (April 2024, Poznan).

        \item K. Nalewajko, \textbf{M. Kapusta}, A. Janiuk. "Initialization of magnetic flux eruptions at accreting black holes"\\
                European Astronomical Society meeting 2023 (poster)
%        \item F. Kayanikhoo, \textbf{M. Kapusta}, M. Cemeljic. \href{https://arxiv.org/abs/2305.03055}{"The maximum mass and deformation of rotating strange quark stars with strong magnetic fields"}\\
%        Submitted to Physical Review D (\#Arxiv 2305.03055 )
        \item F. Kayanikhoo, \textbf{M. Kapusta}, M. Cemeljic. "The maximum gravitational mass and deformation of magnetized rotating strange quark stars" \\
        European Astronomical Society meeting 2023 (poster)

    \end{itemize}
    
    \section{AWARDS \& SCHOLARSHIPS}
    
    \begin{itemize}
        \item Gold medal at 2021 University Physics Competition (as part of the team representing Faculty of Physics)
        \item Silver medal at 1st Global e-Competition on Astronomy and Astrophysics (in place of 14th International Olympiad on Astronomy and Astrophysics), 2020
        \item Bronze medal at 13th International Olympiad on Astronomy and Astrophysics, 2019 Hungary
        \item Winner of 62th and 63th Polish Astronomy Olympiad 
        \item Finalist of 67th and 69th (11th place) Polish Physics Olympiad
        \item Finalist of 70th Polish Mathematical Olympiad
        \item Minister of Education’s scholarship in the year 2018/2019, 2019/2020
        \item Rector scholarship in the academic year 2020/2021, 2021/2022, 2022/2023, 2023/2024
    \end{itemize}

\section{OUTREACH}
    \begin{itemize}
        \item Judge at International Math Competition Naboj (March, 2024)
        \item Judge at 16th International Olympiad on Astronomy and Astrophysics (August, 2023)
        \item Judge at International Math Competition Naboj (March, 2023)

    \end{itemize}

\section{COMPUTER \\ SKILLS} 

{\sl Languages \& Software:} 

%Python (including various libraries like Astropy, SciPy, Matplotlib, Numpy, Pandas, TensorFlow and more, knowledge of Python C API, knowledge of asynchronous and parallel programming), Julia, Fortran, C/C++, knowledge of parallel programming with OpenMP and MPI (Fortran and C/C++ API), knowledge of SIMD instructions,   Gnuplot, IRAF  \\
Computer languages:
\begin{itemize}
    \item Python - advanced
    \item C/C++ - advanced
    \item Julia - advanced
    \item R - intermediate
    \item Fortran - intermediate
    \item Rust - intermediate
\end{itemize}
Additional computer-related skills:
\begin{itemize}
    \item Bayesian modeling using emcee, Pyro, NumPyro, PyMC, and Tensorflow Probability
    \item 3D visualizations with Paraview and Mayavi 
    \item Parallel programming using MPI/OMP
    \item Low-level programming using Python C API
    \item Deep learning experience using PyTorch, Tensorflow, Jax
    \item Grain level parallelism using SIMD
    \item Experience with astronomical Python libraries like Astropy, Astroquery, PyVO
\end{itemize} 
%For code portfolio please see my \href{https://github.com/Wesenheit}{github profile}.\\ \par
%\vspace{0.cm} 
{\sl Operating Systems:} Advanced knowledge of Linux OS


%----------------------------------------------------------------------------------------
%-------------------------------------------------------------------------------------
\section{Language}
\begin{itemize}
    \item English - C1 (103 TOEFL)
    \item Polish - native speaker
    \item German - A2/B1
\end{itemize}
%----------------------------------------------------------------------------------------

\end{resume}
\end{document}
