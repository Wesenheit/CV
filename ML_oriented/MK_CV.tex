\documentclass[a4paper,12pt]{article}
\usepackage{url}
\usepackage[numbers]{natbib}
\usepackage[colorlinks=true, citecolor=blue]{hyperref}
\usepackage{parskip}
\RequirePackage{color}
\RequirePackage{graphicx}
\usepackage[usenames,dvipsnames]{xcolor}
\usepackage{enumitem}
\usepackage[scale=0.9]{geometry}
\usepackage{tabularx}
\newcolumntype{C}{>{\centering\arraybackslash}X}
\usepackage{supertabular}
\newlength{\fullcollw}
\setlength{\fullcollw}{0.47\textwidth}
\usepackage{titlesec}
\usepackage{multicol}
\usepackage{multirow}
\titleformat{\section}{\Large\scshape\raggedright}{}{0em}{}[\titlerule]
\titlespacing{\section}{0pt}{10pt}{10pt}
\definecolor{linkcolour}{rgb}{0,0.2,0.6}
\hypersetup{colorlinks,breaklinks,urlcolor=linkcolour,linkcolor=linkcolour}
\usepackage{fontawesome5}
% job listing environments
\newenvironment{jobshort}[2]
    {
    \begin{tabularx}{\linewidth}{@{}l X r@{}}
    \textbf{#1} & \hfill &  #2 \\[3.75pt]
    \end{tabularx}
    }
    {
    }

\newenvironment{joblong}[2]
    {
    \begin{tabularx}{\linewidth}{@{}l X r@{}}
    \textbf{#1} & \hfill &  #2 \\[3.75pt]
    \end{tabularx}
    \begin{minipage}[t]{\linewidth}
    \begin{itemize}[nosep,after=\strut, leftmargin=1em, itemsep=3pt,label=--]
    }
    {
    \end{itemize}
    \end{minipage}
    }



\begin{document}

\pagestyle{empty}


\begin{tabularx}{\linewidth}{@{} C @{}}
\Huge{Mateusz Kapusta} \\[7.5pt]
\href{https://github.com/username}{\raisebox{-0.05\height}\faGithub\ Wesenheit} \ $|$ \
\href{https://mysite.com}{\raisebox{-0.05\height}\faGlobe \ wesenheit.github.io} \ $|$ \
\href{mailto:email@mysite.com}{\raisebox{-0.05\height}\faEnvelope \ mr.kapusta@student.uw.udu.pl} \ $|$ \
\href{tel:+000000000000}{\raisebox{-0.05\height}\faMobile \ +48 530 510 849} \\
\end{tabularx}

%Interests/ Keywords/ Summary
\section{Summary}
I am a physicist and machine learning researcher involved in disciplinary research across
physics/astronomy and computer science. I am mostly interested in advanced inference techniques and
their potential application in the field of natural sciences.
%Experience
\section{Work Experience}

\begin{joblong}{Machine Learning Research Intern — ASML, Veldhoven, NL}{Oct 2025 – June 2026}
\item Developed advanced Bayesian models for diagnostics of ASML's EUV lithography systems.
\item Integrated differentiable physical simulators with machine learning models to
    enhance state-of-the-art physical modeling through learned corrections.
\item Investigated diverse inference techniques for fast and reliable model inference,
    ranging from traditional Bayesian methods to modern deep learning approaches.
\item Research on efficient multi-GPU solutions for large-scale inference on HPC clusters.
\end{joblong}


\begin{joblong}{Quantitative Developer Intern - UBS AG}{Jul 2025 - September 2025}
\item Working with the C++ high-performant Monte Carlo code used to value Lombard loans.
\item Performance benchmarking, memory and speed optimizations.
\item Reduced processing time $3$ times and reduced required memory $2$ times.
\end{joblong}

\begin{jobshort}{Research Assistant - University of Warsaw}{June 2022 - July 2025}
Performing basic research in Astrophysical sciences (exoplanetary research, statistical methods in astronomy),
written two first-pearson papers and one co-author paper. Research included:
\begin{itemize}[nosep,after=\strut, leftmargin=1em, itemsep=3pt,label=--]
    \item applied research in astronomy, utilizing Bayesian methods, performing statistical
        analysis \citep{ffp,dormant,depth},
    \item imaging techniques on astronomical detectors \citep{ffp}.
\end{itemize}
\end{jobshort}
%Projects
\section{Projects}

\begin{tabularx}{\linewidth}{ @{}l r@{} }
\textbf{Neural Posterior Estimation for Spectral Energy Distribution fitting} & \hfill \href{https://github.com/Wesenheit/Iris-ML}{Github} \\[3.75pt]
\multicolumn{2}{@{}X@{}}{Reaserching applications of Neural Posterior Estimation in Astronomy. Spectral Energy Distribution fitting is powerfull yet
computationally expensive technique widelly used in Astronomy. I demonstrated that with the help of normalizing flows one can
speed up the inference of parameters by several orders of magnitudes.
To do so, I designed my own complex preprocessing model that was paried with Masked Autoregressive Flow.
Model was then fined-tuned so it can operate on real-life astronomical data. \textbf{Model was accepted as
poster at the ICML 2025 co-located workshop: Machine Learning for Astrophysics} \citep{irisML}.}  \\
\end{tabularx}

\begin{tabularx}{\linewidth}{ @{}l r@{} }
\textbf{Computational Fluid Dynamics on GPU-s} & \hfill \href{https://github.com/Wesenheit/Verona}{Github} \\[3.75pt]
\multicolumn{2}{@{}X@{}}{Astrophysical research heavilly rellies on the simulations. In the project, a Riemann
solver was used to solve the equations of relativistic fluids, which are very common in astrophysical research.
It is written in CUDA-enabled Julia code
paralellized with MPI to utilize massive GPU clusters.
It relies on manually written computational kernels that are optimized for A100 NVIDIA GPU-s
(compute - memory bandwidth - communication).
Project is being developed with the help of
EuroHPC development access computing grant EHPC-DEV-2025D02-085 on the Leonardo BOOSTER, where I am a Co-PI.}  \\
\end{tabularx}


%---------------------------------------------------------------------------------- ------
%	EDUCATION
%----------------------------------------------------------------------------------------
\section{Education}
\begin{tabularx}{\linewidth}{@{}l X@{}}

2020 - 2025 & Masters's Degree at \textbf{University of Warsaw} \hfill (GPA: 4.90/5.00) \newline
Master's in Astrophysics with minor in Computer Science at the University of Warsaw as part of Interdysciplinary Collage of
Interdysciplinary studies.

Specialization: Machine Learning, High Performance Computing
\end{tabularx}

%----------------------------------------------------------------------------------------
%	PUBLICATIONS
%----------------------------------------------------------------------------------------
\bibliographystyle{aasjournal} % or apj, aasjournal, etc.
\bibliography{MK}
%----------------------------------------------------------------------------------------
%	SKILLS
%----------------------------------------------------------------------------------------
\section{Awards \& Scholarships}
  \begin{itemize}[nosep,after=\strut, leftmargin=1em, itemsep=3pt,label=--]
      \item Gold medal at 2021 University Physics Competition (as part of the team representing Faculty of Physics)
      \item Silver medal at the 1st Global e-Competition on Astronomy and Astrophysics (in place of 14th International Olympiad on Astronomy and Astrophysics), 2020
      \item Bronze medal at 13th International Olympiad on Astronomy and Astrophysics, 2019 Hungary
      \item Winner of 62th and 63th Polish Astronomy Olympiad for high school students
      \item Finalist of 67th and 69th (11th place) Polish Physics Olympiad for high school students
      \item Finalist of 70th Polish Mathematical Olympiad for high school students
      \item Minister of Education's scholarship in the year 2018/2019, 2019/2020
      \item Rector's scholarship in the academic year 2020/2021, 2021/2022, 2022/2023, 2023/2024, 2024/2025
  \end{itemize}

    \section{OUTREACH}
  \begin{itemize}[nosep,after=\strut, leftmargin=1em, itemsep=3pt,label=--]
      \item Jury at the international math competition Naboj (March, 2025)
      \item Jury at the international math competition Naboj (March, 2024)
      \item Judge at the 16th International Olympiad on Astronomy and Astrophysics (August, 2023)
      \item Jury at the international math competition Naboj (March, 2023)
  \end{itemize}

\section{Skills}
\begin{tabularx}{\linewidth}{@{}l X@{}}
    ML related libraries &  \normalsize{PyTorch, Jax, Pyro, NumPyro, DVC, CUDA \& MPI}\\
    Programing Languages &  \normalsize{Python, C++, Julia, Go}\\
\end{tabularx}

\vfill
\center{\footnotesize Last updated: \today}

\end{document}
