\documentclass[english,a4paper]{europasscv}
\usepackage[english]{babel}

\ecvname{Mateusz Kapusta}
\ecvtelephone{+48 530 510 849}

\ecvemail{mr.kapusta@student.uw.edu.pl}
\ecvhomepage{https://wesenheit.github.io/}
\ecvgithubpage{www.github.com/Wesenheit}
% \ecvgitlabpage{www.gitlab.com/smith}
% \ecvlinkedinpage{www.linkedin.com/in/katie-smith}
\ecvorcid[label, link]{0009-0005-6812-5605}
%\ecvim{AOL Messenger}{katie.smith}
%\ecvim{Google Talk}{ksmith}

% \ecvgender{Female}
\ecvdateofbirth{14 July 2001}
\ecvnationality{Polish}

 \ecvpicture[width=3.8cm]{MK_photo}

% \date{}

\begin{document}
  \begin{europasscv}

  \ecvpersonalinfo

  \ecvsection{Summary}
  \ecvitem{}{
  I am a Master's student, currently studying at the Faculty of Physics, University of Warsaw. I completed my Bachelor's thesis under the supervision of  Przemysław Mróz at the 
  Warsaw University Observatory. I graduated from the college of Inter-Faculty Individual Studies, where I divided my time between the Faculty of Physics (FUW)
  and Faculty of Mathematics, Informatics and Mechanics (MIMUW).
  I am passionate about the interdisciplinary research, where physical/astronomical problems can be approach from mathematical/computational perspective. 
  Especially interested in Bayesian Modelling and High Performance Computing.
  During my studies I had an opportunity to work on different aspects of Astrophysics, ranging from observational projects to 
  theoretical ones.
  In my free time I like exploring machine learning based methods applied in the field of Natural Sciences,
  solving problems across different disciplines from Bioinformatics to Astronomy.}
  %Currently I work with Krzysztof Nalewajko (Nicolaus Copernicus Astronomical Center - NCAC), investigating astrophysical jets launched from 
  %Supermassive black holes with the help of numerical simulations.
  \ecvsection{Work experience}

  \ecvtitle{April 2024 - present}{Student research assistant, Nicolaus Copernicus Astronomical Center}
  \ecvitem{}{Working under the supervision of Krzysztof Nalewajko (NCAC).}
  \ecvitem{}{Project in cooperation with Bart Ripperda (CITA, Toronto) and Alexander Philippov (University of Maryland).}
  \ecvitem{}{Project is planned to be presented in the Master's thesis.}
  \ecvitem{}{\begin{ecvitemize}
      \item Working with extreme resolution magnetohydrodynamical (MHD) simulations created with H-AMR code.
      \item Studying stability of astrophysical jets launched from black holes.
      \item Investigating influence of magnetic flux eruptions, associated with Magnetically Arrested Discs (MAD), on the properties of the jet. 
  \end{ecvitemize}}
  
  \ecvtitle{November 2022 - present}{Student research assistant at Astronomical Observatory}
  \ecvitem{}{Student position in grant 2021/41/B/ST9/00252, working under the supervision of Przemysław Mróz.}
  \ecvitem{}{\begin{ecvitemize}
      \item Performing MCMC modelling of microlensing events discovered as the part of $4$th phase of the OGLE project.
      \item Investigating Free Floating Planet (FFP) microlensing event OGLE-2023-BLG-0524. Working on theoretical modelling, analyzing legacy Hubble Space Telescope (HST)
      photometry, performing detectability simulations in order to verify FFP hypothesis. Project involves a significant data-related part including Bayesian modelling,
      working with real PSF-s and more.
      \item Research resulted in 4th-author publication in ApJS, currently preparing first author publication about the OGLE-2023-BLG-0524 event.
  \end{ecvitemize}}
  
  \ecvtitle{November 2023 - present}{Research project within the field of Deep Learning}
  \ecvitem{}{Working on deep learning model aimed to speed up MCMC inference of Spectral Energy Distribution (SED) fitting (written in PyTorch).
  Project is centered on the analysis of the photometric data to determine physical parameters of stars (temperature, metalicty, reddening, etc.).
  Python-based PyTorch framework has been used to implement the networks. }
  \ecvitem{}{\begin{ecvitemize}
      \item Designed and coded a deep neural network based on the transformer architecture, allowing one to process any number of photometric measurements.
      \item Implemented a neural network for Neural Posterior Estimation (NPE, based on the Masked Autoregressive Flow model) to estimate the posterior of Bayesian model.
      \item Project resulted in two posters presented on international machine learning conferences (ML in PL 2024 and GHOST 2024).
  \end{ecvitemize}
  }

  \ecvtitle{November 2024 - present}{Research project at the Astronomical Observatory, University of Warsaw}
  \ecvitem{}{Developing GPU-accelerated, mutli-node special-relativistic hydrodynamical code (writen in Julia).
  Code utilizes the Riemann solver and allows for various reconstuction methods (Minmod, Weno-Z, PPM).
  Project is done under the supervision of prof. Tomasz Bulik.}
  \ecvitem{}{\begin{ecvitemize}
      \item Writing custom CUDA kernels to maximize the performance on Ampere-based Nvidia GPU-s.
      \item Profiling the code with respect to different bandwidths (computational - memory - communication).
      \item Checking the performance of different reconstruction methods, trying to pinpoint most suitable algorithm for given GPU architecture.
      \item Project is being developed with the help of EuroHPC development access computing grant EHPC-DEV-2025D02-085 on the Leonardo BOOSTER, where I am a Co-PI.
  \end{ecvitemize}
  }




  \ecvtitle{July 2022 - July 2023}{Bachelor's Thesis at Warsaw University Astronomical Observatory}
  \ecvitem{}{Working under the supervision of Przemysław Mróz on the data analysis from the OGLE survey to search for dormant black hole candidates.}
  \ecvitem{}{
  \begin{ecvitemize}
      \item Analysing OGLE data using the method introduced in Gomel et al. \href{https://arxiv.org/abs/2008.11209}{2021}.
      \item Designing Python based MCMC code to assemble spectral energy distribution (SED) for candidate objects.
      \item Inference of the parameters of binaries using the OGLE and Gaia DR3 data, searching for compact companion stars.
      \item Project resulted in first-author paper accepted in Acta Astronomica.
  \end{ecvitemize}}

  \ecvtitle{July 2022 - October 2022}{Intern at Nicolaus Copernicus Astronomical Center}
  \ecvitem{}{Project: "Measuring the structure of relativistic jets in numerical simulation results" under the supervision of Krzysztof Nalewajko (NCAC) and Agnieszka Janiuk (CFT PAN).}
  \ecvitem{}{
  \begin{ecvitemize}
      \item Worked with results from HARM MHD code to study the structure of magnetically arrested discs.
      \item Developed a few Python routines to search for magnetic reconnection and other interesting magnetic phenomena.
      \item Work accomplished during the internship resulted in second-author publication submitted to Astronomy \& Astrophysics (in revision).
  \end{ecvitemize}}

  \ecvtitle{July 2021 - October 2023}{Intern at Nicolaus Copernicus Astronomical Center}
  \ecvitem{}{Project: "Energy of a Strange Quark Star" under the supervision of Fatemeh Kayanikhoo, M. Cemeljic and J. Zdunik}
  \ecvitem{}{\begin{ecvitemize}
    \item Worked with LORENE library to study the structure of relativistic strange quark stars, ported part of functions to work with 
    C++17 standard and MPI multithread environment.
    \item Developed Python code to calculate the external energy of a star contained in a magnetic field.
    \item Developed multi-threaded C++ code to calculate the equation of state of the magnetized strange matter.
    \end{ecvitemize}}

  \ecvsection{Education and training}    
  \ecvtitle{2023-2025(expected)}{Master of Science in Astrophysics}
  \ecvitem{}{Faculty of Physics, University of Warsaw, Poland}
  \ecvitem{}{Thesis: TBD}
  \ecvitem{}{Supervisor: Krzysztof Nalewajko (email: knalew@camk.edu.pl)}
  \ecvitem{}{Major: Astronomy}

  \ecvtitle{2020-2023}{Bachelor of Science in Astrophysics}
  \ecvitem{}{College of Inter-faculty Studies in Mathematics and Natural Sciences, University of Warsaw, Poland. My studies were divided between the Faculty of Physics and 
  Faculty of Mathematics, Informatics and Mechanics.}
  \ecvitem{}{Thesis: \textbf{Search for dormant black holes in the OGLE data}}
  \ecvitem{}{Grade:  4.96/5 (2-5 scale, 5 is the highest), graduated with distinctions}
  \ecvitem{}{Supervisor: Przemysław Mróz (email: pmroz@astrouw.edu.pl)}
  \ecvitem{}{Major: Astronomy}
  \ecvitem{}{Minor: Mathematics, Computer Science}
    
  \ecvsection{Personal skills}
  \ecvmothertongue{Polish}
  \ecvitem{}{English - C1 (103 TOEFL)}
  \ecvitem{}{German - B1 }

  \ecvsection{Publications}
  \ecvitem{}{
    \begin{ecvitemize}
      \item K. Nalewajko, \textbf{M. Kapusta}, A. Janiuk "Chaotic Magnetic Disconnections Trigger Flux Eruptions in Accretion Flows Channeled onto Magnetically Saturated Kerr Black Holes" 
      Accepted in Astronomy \& Astrophysics \texttt{[arXiv:2410.08280]}
      \item P. Mróz, A. Udalski, M. Szymański, \textbf{M. Kapusta}, et al. "Microlensing Optical Depth and Event Rate toward the Large Magellanic Cloud Based on 20 yr of OGLE Observations"
      ApJS, 273, 4 (2024) \texttt{[arXiv:2403.02398]}
      \item \textbf{M. Kapusta}, P. Mróz. "The search for Dormant Black Holes in the OGLE data"\\ 2023, Acta Astron., 73, 197 \texttt{[arXiv:2401.11293]}
      
      
      %\item F. Kayanikhoo, \textbf{M. Kapusta}, M. Cemeljic.{"The maximum mass and deformation of rotating strange quark stars with strong magnetic fields"}\\
      %Submitted to Physical Review D \texttt{[arXiv:2305.03055]}
  \end{ecvitemize}}

  \ecvsection{Talks \& Posters}
  \ecvitem{}{\begin{ecvitemize}
      \item \textbf{M. Kapusta}, K. Nalewajko, B. Ripperda, A. Philippov "3D geometry and magnetic connections of the erupting black-hole jet", oral presentation at the conference "Feeling the pull and the pulse of relativistic magnetospheres", Les Houches April 2025.
      \item \textbf{M. Kapusta} "Extreme resolution GRMHD simulations of Astrophysical jets", presentation at 10th Symposium for Young Researchers, presentation awarded with distinctions (September 2024, Warsaw)
      \item \textbf{M. Kapusta} "Iris-ML: Neural Posterior Estimation for the Spectral Energy Distribution fitting." poster at ML in PL 2024 machine learning conference (November 2024, Warsaw).
  \end{ecvitemize}}
  

    
  \ecvsection{Awards \& Scholarships}
  \ecvitem{}{ 
  \begin{ecvitemize}
      \item Gold medal at 2021 University Physics Competition (as part of the team representing Faculty of Physics)
      \item Silver medal at the 1st Global e-Competition on Astronomy and Astrophysics (in place of 14th International Olympiad on Astronomy and Astrophysics), 2020
      \item Bronze medal at 13th International Olympiad on Astronomy and Astrophysics, 2019 Hungary
      \item Winner of 62th and 63th Polish Astronomy Olympiad for high school students
      \item Finalist of 67th and 69th (11th place) Polish Physics Olympiad for high school students
      \item Finalist of 70th Polish Mathematical Olympiad for high school students
      \item Minister of Education's scholarship in the year 2018/2019, 2019/2020
      \item Rector's scholarship in the academic year 2020/2021, 2021/2022, 2022/2023, 2023/2024, 2024/2025
  \end{ecvitemize}}

    \ecvsection{OUTREACH}
    \ecvitem{}{ 
  \begin{ecvitemize}
      \item Jury at the international math competition Naboj (March, 2025)
      \item Jury at the international math competition Naboj (March, 2024)
      \item Judge at the 16th International Olympiad on Astronomy and Astrophysics (August, 2023)
      \item Jury at the international math competition Naboj (March, 2023)
  \end{ecvitemize}}

  \ecvsection{Computer \\ Skills} 

  \ecvitem{}{ Languages \& Software:} 
  
  %Python (including various libraries like Astropy, SciPy, Matplotlib, Numpy, Pandas, TensorFlow and more, knowledge of Python C API, knowledge of asynchronous and parallel programming), Julia, Fortran, C/C++, knowledge of parallel programming with OpenMP and MPI (Fortran and C/C++ API), knowledge of SIMD instructions,   Gnuplot, IRAF  \\
  \ecvitem{}{Computer languages:}
  \ecvitem{}{\begin{ecvitemize}
      \item Python - advanced
      \item C/C++ - advanced
      \item Julia - advanced
      \item Fortran - intermediate
      \item Rust - intermediate
  \end{ecvitemize}}
  \ecvitem{}{  Additional computer-related skills:}
  \ecvitem{}{\begin{ecvitemize}
    \item Advanced knowledge of Linux OS
    \item Bayesian modeling using emcee, Pyro/NumPyro, PyMC, blackjax and Tensorflow Probability
    \item 3D visualizations with Paraview and Mayavi 
    \item Parallel programming using MPI/OMP, GPGPU programming in CUDA
    \item Low-level programming using Python C API
    \item Deep learning experience using PyTorch, Tensorflow, Jax
    \item Low-level parallelism using SIMD
    \item Experience with astronomical Python libraries like Astropy, Astroquery, PyVO
    \end{ecvitemize}}

  \end{europasscv}

\end{document}
