
\documentclass[margin, 10pt]{res} % Use the res.cls style, the font size can be changed to 11pt or 12pt here
\usepackage{tabularx}

\usepackage{helvet} % Default font is the helvetica postscript font
%\usepackage{newcent} % To change the default font to the new century schoolbook postscript font uncomment this line and comment the one above
\usepackage{fontawesome}% <=============================================
\usepackage{hyperref}
\usepackage{wrapfig}
\usepackage{graphicx}
\setlength{\textwidth}{5.5in} % Text width of the document

\begin{document}

%----------------------------------------------------------------------------------------
%	NAME AND ADDRESS SECTION
%----------------------------------------------------------------------------------------

\moveleft.5\hoffset\centerline{\large\bf Mateusz Kapusta} % Your name at the top

\moveleft\hoffset\vbox{\hrule width\resumewidth height 1pt}\smallskip % Horizontal line after name; adjust line thickness by changing the '1pt'

\moveleft.5\hoffset\centerline{Astronomical Observatory, University of Warsaw, Al. Ujazdowskie 4, 00-478 Warsaw,Poland}
\moveleft.5\hoffset\centerline{\href{https://github.com/Wesenheit}{\faGithubSquare https://github.com/Wesenheit}}
\moveleft.5\hoffset\centerline{\href{https://wesenheit.github.io/}{Website: https://wesenheit.github.io/}}
%----------------------------------------------------------------------------------------

\begin{resume}

\section{EDUCATION}

{\sl Master of Science, \textbf{2023-2026} (expected)}\\
Master of Science in Astrophysics\\
Warsaw University Observatory, Faculty of Physics\\

{\sl Bachelor of Science, \textbf{2020-2023}}\\
Inter-faculty Individual Studies in Mathematics and Natural Sciences, University of Warsaw, Poland\\
Grade:  4.96/5 (2-5 scale, 5 is highest), graduated with distinctions\\
Major: Astrophysics, Physics\\
Minor: Computer Science \par

\section{EXPERIENCE}


{\sl Machine Learning Intern, ASML (Veldhoven, Netherlands)} \hfill October 2025 -- June 2026\\
Working in the R\&D department on machine learning solutions for diagnosing degradation in NXE lithography machines (various contaminations in the optical column).
\begin{itemize}
    \item Researched hybrid models integrating physical simulations with machine learning, focusing on data-driven corrections to the physical models.
    \item Implemented Bayesian inference algorithms to infer contamination profiles in the machines based on the hybrid models.
    \item Developed high-performance implementations of inference algorithms, optimized for both edge devices (ordinary laptops) and large-scale HPC clusters.
\end{itemize}

{\sl Quantitative Developer Intern, UBS AG} \hfill July 2025 -- September 2025\\
Worked on a high-performance, multi-code C++ engine used to price loans within the Lombard Lending unit.
\begin{itemize}
    \item Performed code benchmarking and bottleneck analysis to identify performance issues.
    \item Investigated parallel efficiency and load balancing using Intel TBB.
    \item Applied memory and computational optimizations, resulting in a codebase that was 3× faster and 2× more memory-efficient.
\end{itemize}



{\sl Student research assistant, Nicolaus Copernicus Astronomical Center} \hfill April 2023 - July 2025\\
Working under supervision of Krzysztof Nalewajko (NCAC).
Project in cooperation with Bart Ripperda (CITA, Toronto) and Alexander Philippov (University of Maryland).
\begin{itemize}
    \item Working with extreme resolution Magnetohydrodynamical (MHD) simulations created with H-AMR code.
    \item Studying stability of a astrophysical jets launched from black holes.
    \item Investigating influence of magnetic flux eruptions, associated with Magnetically Arrested Discs (MAD), on the properties of the jet.
    \item Presented work will be submitted as a second-author paper to Astrophysical Journal Letters.
\end{itemize}


{\sl Student research assistant at Astronomical Observatory} \hfil November 2022 - July 2025\\
Student position in grant 2021/41/B/ST9/00252, working under the supervision of Przemysław Mróz.
\begin{itemize}
    \item Performing MCMC modelling of microlensing events discovered as the part of $4$th phase of the OGLE project.
    \item Investigating Free Floating Planet (FFP) microlensing event OGLE-2023-BLG-0524. Working on theoretical modelling, analysing legacy Hubble Space Telescope (HST)
    photometry, performing detectability simulations in order to verify FFP hypothesis.
    \item Research resulted in 4th-author publication in ApJS, and first-author publication in A\&A about OGLE-2023-BLG-0524 event.
\end{itemize}



{\sl Bachelor's Thesis at Warsaw University Astronomical Observatory} \hspace*{0.1cm} July 2022 - July 2023\\
Working under the supervision of dr Przemysław Mróz on the data analysis from the OGLE survey to search for Dormant Black Hole candidates.
\begin{itemize}
    \item Analysing OGLE data using the method introduced in Gomel et al. \href{https://arxiv.org/abs/2008.11209}{2021}
    \item Designing Python based MCMC code to assemble spectral energy distribution (SED) for candidate objects.
    \item Inference of the parameters of binaries using the OGLE and Gaia DR3 data, searching for compact companion stars.
    \item Project resulted in first-author paper accepted in Acta Astronomica.
\end{itemize}



{\sl Intern at Nicolaus Copernicus Astronomical Center}\hfill July 2022 - October 2022\\
Project: "Measuring the structure of relativistic jets in numerical simulation results" under the supervision of Krzysztof Nalewajko (NCAC) and prof. Agnieszka Janiuk (CFT PAN).
\begin{itemize}
    \item Worked with results from HARM MHD code to study the structure of magnetically arrested discs.
    \item Developed a few Python routines to search for magnetic reconnection and other interesting magnetic phenomena.
    \item Work accomplished during the internship resulted in second-author publication submitted to Astronomy \& Astrophysics.
\end{itemize}

{\sl Intern at Nicolaus Copernicus Astronomical Center}\hfill July 2021 - October 2023\\
Project: "Energy of a Strange Quark Star" under the supervision of Fatemeh Kayanikhoo  and dr M. Cemeljic
\begin{itemize}
    \item Worked with LORENE library to study the structure of relativistic strange quark stars, ported part of functions to work with
    C++17 standard and MPI multithread environment.
    \item Developed Python code to calculate the external energy of a star contained in a magnetic field.
    \item Developed multi-threaded C++ code to calculate the equation of state of the magnetized strange matter.
\end{itemize}

\section{PROJECTS}
\begin{tabularx}{\linewidth}{ @{}l r@{} }
\textbf{Neural Posterior Estimation for Spectral Energy Distribution fitting} & \hfill \href{https://github.com/Wesenheit/Iris-ML}{Github} \\[3.75pt]
\multicolumn{2}{@{}X@{}}{Reaserching applications of Neural Posterior Estimation in Astronomy. Spectral Energy Distribution fitting is powerfull yet
computationally expensive technique widelly used in Astronomy. I demonstrated that with the help of normalizing flows one can
speed up the inference of parameters by several orders of magnitudes.
To do so, I designed my own complex preprocessing model that was paried with Masked Autoregressive Flow.
Model was then fined-tuned so it can operate on real-life astronomical data. Model was accepted as
poster at the ICML 2025 co-located workshop: Machine Learning for Astrophysics.}  \\
\end{tabularx}

\begin{tabularx}{\linewidth}{ @{}l r@{} }
\textbf{Computational Fluid Dynamics on GPU-s} & \hfill \href{https://github.com/Wesenheit/Verona}{Github} \\[3.75pt]
\multicolumn{2}{@{}X@{}}{Astrophysical research heavilly rellies on the simulations. In the project, a Riemann
solver was implemented to solve the equations of relativistic fluids.
It is written in CUDA-enabled Julia code
paralellized with MPI to utilize massive GPU clusters.
It relies on manually written computational kernels that are optimized for A100 NVIDIA GPU-s
(compute - memory bandwidth - communication).
Project is being developed with the help of
EuroHPC development access computing grant EHPC-DEV-2025D02-085 on the Leonardo BOOSTER, where I am a Co-PI.}  \\
\end{tabularx}


\section{PUBLICATIONS}
    \begin{itemize}
        \item \textbf{M. Kapusta}, P. Mroz, et al. "HST pre-imaging of a free-floating planet candidate microlensing event"
        Accepted in Astronomy \& Astrophysics \texttt{[arXiv:2507.01109]}
        \item K. Nalewajko, \textbf{M. Kapusta}, A. Janiuk "Chaotic Magnetic Disconnections Trigger Flux Eruptions in Accretion Flows Channeled onto Magnetically Saturated Kerr Black Holes"
        Accepted in Astronomy \& Astrophysics \texttt{[arXiv:2410.08280]}
        \item P.Mróz, A.Udalski, M.Szymański, \textbf{M. Kapusta}, et al. "Microlensing Optical Depth and Event Rate toward the Large Magellanic Cloud Based on 20 yr of OGLE Observations"
        Accepted in ApJS \texttt{[arXiv:2403.02398]}
        \item \textbf{M. Kapusta}, P. Mróz. "The search for Dormant Black Holes in the OGLE data"\\ Accepted in Acta Astronomica \texttt{[arXiv:2401.11293]}

    \end{itemize}

\section{TALKS \& POSTERS}
    \begin{itemize}
        \item \textbf{M. Kapusta}: "IrisML: Neural Posterior Estimation for the Spectral Energy Distribution fitting"
        ICML 2025: Machine learning for Astrophysics workshop.
    \end{itemize}

    \section{AWARDS \& SCHOLARSHIPS}

    \begin{itemize}
        \item Gold medal at 2021 University Physics Competition (as part of the team representing Faculty of Physics)
        \item Silver medal at 1st Global e-Competition on Astronomy and Astrophysics (in place of 14th International Olympiad on Astronomy and Astrophysics), 2020
        \item Bronze medal at 13th International Olympiad on Astronomy and Astrophysics, 2019 Hungary
        \item Winner of 62th and 63th Polish Astronomy Olympiad
        \item Finalist of 67th and 69th (11th place) Polish Physics Olympiad
        \item Finalist of 70th Polish Mathematical Olympiad
        \item Minister of Education’s scholarship in the year 2018/2019, 2019/2020
        \item Rector scholarship in the academic year 2020/2021, 2021/2022, 2022/2023, 2023/2024
    \end{itemize}

\section{OUTREACH}
    \begin{itemize}
        \item Judge at International Math Competition Naboj (March, 2023-2025)
        \item Judge at 16th International Olympiad on Astronomy and Astrophysics (August, 2023)

    \end{itemize}

\section{COMPUTER \\ SKILLS}

{\sl Languages \& Software:}
\begin{itemize}
    \item Computer languages - Python,C/C++, Julia, R, Rust, Fortran
    \item Deep learning knowledge using PyTorch/JAX.
    \item Parallel programming using MPI/OMP.
    \item GPU programming skills using CUDA (also paired with MPI).
\end{itemize}


%----------------------------------------------------------------------------------------
%-------------------------------------------------------------------------------------
\section{Language}
\begin{itemize}
    \item English - C1 (103 TOEFL)
    \item Polish - native speaker
    \item German - A2/B1
\end{itemize}
%----------------------------------------------------------------------------------------

\end{resume}
\end{document}
